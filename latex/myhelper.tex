\documentclass{article}
\usepackage{amsmath,amsfonts,amssymb}
\usepackage{array}
\usepackage{float}
\usepackage{multirow}
\usepackage{amsthm}
\usepackage{amsbsy}
\usepackage{amstext}
\usepackage{amssymb}
\numberwithin{equation}{subsection}
\usepackage{graphicx}
\title{Thoughts along developing myHelper}
\author{jdong}
\begin{document}
	\maketitle
	\abstract{
		I'm going to write this library for my own use but with a long term plan such that the generality level is no less than one of a shared library. 
		Until 2017-02-05, the rough skeleton is thought to be a CUDA-C/C++ dynamic library compilable on both unix and windows and on windows using Excel as the interactive GUI.
	}
	\part{Top-Down}
	This part starts with Excel programming, moving downwards the center of an XLL that stores CUDA-C programs.
	\section{Excel UI}
	\section{XLL Building}
	
	\part{Bottom-Up}
	This part starts with CUDA programming, moving upwards the center of a DLL that can be called by Excel's C API.
	\section{GPU Hardware}
	\section{CUDA Programming}
\end{document}
