\documentclass{article}
\usepackage{amsmath,amsfonts,amssymb}
\usepackage{array}
\usepackage{float}
\usepackage{multirow}
\usepackage{amsthm}
\usepackage{amsbsy}
\usepackage{amstext}
\usepackage{amssymb}
\numberwithin{equation}{subsection}
\usepackage{graphicx}
\usepackage{hyperref}
\usepackage[authoryear]{natbib}
\usepackage{charter}
\usepackage{listings}

\title{Thoughts along developing myHelper}
\author{jdong}
\begin{document}
	\maketitle
	\abstract{
		I'm going to write this library for my own use but with a long term plan such that the generality level is no less than one of a shared library. 
		Until 2017-02-05, the rough skeleton is thought to be a CUDA-C/C++ dynamic library compilable on both unix and windows and on windows using Excel as the interactive GUI.
	}
	\part{Top-Down}
	This part starts with Excel programming, moving downwards the center of an XLL that stores CUDA-C programs.
	\section{Excel UI}
	\section{Excel's C API and XLL Building}
	Tools: Excel SDK page on MSDN: \url{https://msdn.microsoft.com/en-us/library/office/bb687883.aspx}
	Find the files (dropbox/xl):
	\begin{enumerate}
		\item XLCALL.H(1481), XLCALL.CPP(120);
		\item FRAMEWRK.H(71), FRAMEWRK.C(2090)
		\begin{enumerate}
			\item If made copies of those include files to the project directory, the angle bracktes should be changed to quotes.
		\end{enumerate}
		\item MemoryManager.h(58), MemoryManager.cpp(207);
		\item MemoryPool.h(34), MemoryPool.cpp(80);
		\item Generic.sln(with GENERIC.C, GENERIC.H, GENERIC.DEF, RESOURCE.H).
	\end{enumerate}
	Create dllmain.cpp\begin{enumerate}
		\item Defines the XLL function table \cite{BoveyWallentin2009ProfessionalExcelDevelopment-Th}
		A close relative in VBA is the Application.MacroOptions Method \url{https://msdn.microsoft.com/en-us/library/office/ff838997.aspx}. An explanation of the XLL function table can be found in \cite{BoveyWallentin2009ProfessionalExcelDevelopment-Th} pp1036. Following is an example code for just 1 function.
		\begin{lstlisting}
#define rgFuncsRows 1
static LPWSTR rgFuncs[1][11] = {
	{L"myFun",	//name: name of the C function 
	 L"BBB",	//type: <ret><arg1><arg2>
			//A:Boolean(short int), B:double, 
			//D:ByteString(unsigned char*), 
			//I:short int, J:int, K:Array(FP*)
			//P:oper*, R:xloper*
	 L"myFun",	//xlName: name in Excel
	 L"x,y",	//xlArgs: name of args in Excel
	 L"1",		//function macro type: 
			//1=WorksheetFunction; 
			//2=XLM macro sheet function;
			//0=hidden function macro.
	 L"myCategory",	// function category
	 L"^+m",		//shortcut
	 L"",		//help topic ID
	 L"a sample function",	//description in fx box
	 L"describe arg1",	//arg1 description
	 L"describe arg2",	//arg2 description
	}
};
		\end{lstlisting}
		
		\item Holds DLLMain, the api entry. This is common for all DLLs. It is used to hold library initialization code. The function is explained in the following:
		\begin{lstlisting}
//typedef int BOOL
//#define WINAPI __stdcall
//#define APIENTRY WINAPI
//typedef void* LPVOID
//typedef unsigned long DWORD
//#define DECLARE_HANDLE(name) struct name##__{int unused;}; \
//typedef struct name##__ *name  
//## is the preprocessor token concatenation operator
//DECLARE_HANDLE(HINSTANCE);
//typedef HINSTANCE HMODULE
BOOL APIENTRY DllMain(HMODULE hModule,
		DWORD ul_reason_for_call,
		LPVOID lpReserved){
	switch(ul_reason_for_call){
		case DLL_PROCESS_ATTACH:	//1
		case DLL_THREAD_ATTACH:		//2
		case DLL_THREAD_DETACH:		//3
		case DLL_PROCESS_DETACH:	//0
			break;
	}
	return TRUE;	//#define TRUE 1
}
		\end{lstlisting}
		\item 
	\end{enumerate}
	\part{Bottom-Up}
	This part starts with CUDA programming, moving upwards the center of a DLL that can be called by Excel's C API.
	\section{GPU Hardware}
	\section{CUDA Programming}
	
	\bibliographystyle{plainnat}
	\bibliography{MasterBibliography}
\end{document}
