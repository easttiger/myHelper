\documentclass{article}
\usepackage{amsmath,amsfonts,amssymb}
\usepackage{array}
\usepackage{float}
\usepackage{multirow}
\usepackage{amsthm}
\usepackage{amsbsy}
\usepackage{amstext}
\usepackage{amssymb}
\numberwithin{equation}{subsection}
\usepackage{graphicx}
\usepackage{hyperref}
\usepackage[authoryear]{natbib}
\usepackage{charter}
\title{Thoughts along developing myHelper}
\author{jdong}
\begin{document}
	\maketitle
	\abstract{
		I'm going to write this library for my own use but with a long term plan such that the generality level is no less than one of a shared library. 
		Until 2017-02-05, the rough skeleton is thought to be a CUDA-C/C++ dynamic library compilable on both unix and windows and on windows using Excel as the interactive GUI.
	}
	\part{Top-Down}
	This part starts with Excel programming, moving downwards the center of an XLL that stores CUDA-C programs.
	\section{Excel UI}
	\section{Excel's C API and XLL Building}
	Tools: Excel SDK page on MSDN: \url{https://msdn.microsoft.com/en-us/library/office/bb687883.aspx}
	Find the files (dropbox/xl):
	\begin{enumerate}
		\item XLCALL.H(1481), XLCALL.CPP(120);
		\item FRAMEWRK.H(71), FRAMEWRK.C(2090)
		\begin{enumerate}
			\item If made copies of those include files to the project directory, the angle bracktes should be changed to quotes.
		\end{enumerate}
		\item MemoryManager.h(58), MemoryManager.cpp(207);
		\item MemoryPool.h(34), MemoryPool.cpp(80);
		\item Generic.sln(with GENERIC.C, GENERIC.H, GENERIC.DEF, RESOURCE.H).
	\end{enumerate}
	Create dllmain.cpp\begin{enumerate}
		\item Defines the XLL function table \cite{BoveyWallentin2009ProfessionalExcelDevelopment-Th}
		\item Holds DLLMain, the api entry.
		\item 
	\end{enumerate}
	\part{Bottom-Up}
	This part starts with CUDA programming, moving upwards the center of a DLL that can be called by Excel's C API.
	\section{GPU Hardware}
	\section{CUDA Programming}
	
	\bibliographystyle{plainnat}
	\bibliography{MasterBibliography}
\end{document}
